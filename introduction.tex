%!TEX root=index.tex
% establish that there is a business problem and the hypothesis should solve it
% maybe 1-1.5 pages
\newacronym{merra}{MERRA}{Modern-Era Retrospective analysis for Research and Applications}
\newacronym{html}{HTML}{Hyper Text Markup Language}
\newacronym{sme}{SME}{Subject Matter Expert}
\newacronym{pdf}{PDF}{Portable Document Format}

\section{Introduction}
Everywhere we turn, the big data movement is changing how industries are collecting, analyzing, and responding to the data around them. Efforts, by the Korean Meteorological Administration, are underway to upgrade the ability to predict weather patterns and the severity of weather events across the South Korean peninsula. IBM is engaged in similar work in New York City and Rio de Janeiro, in preparation for the 2014 Olympiad, with goals of accurately predicting short term weather\cite{rwe}. What if the vast repositories of weather and climate data could be collected, stored, and analyzed by current distributed processing algorithms to produce climate predictions months, not weeks, in advance?\\

\textsc{CSC's} original \climatedge  service offered forward looking reports  to commodities markets. The reports were compiled with a cursory qualitative analysis of the NASA \gls{merra}  data with commentary by subject matter experts in climate sciences. The monthly reports included Global Agriculture, Global Energy, Sugar and Soft Commodities, Grain and Oilseeds, and Energy/Natural Gas\cite{climatedgeurl}. Interviews conducted with various people associated with the original [and current] \climatedge offering hinted that the product was not marketed well by \textsc{CSC} and subsequently, did not fare well.\\

The following equation can be applied to the general insurance industry business model: 
\begin{equation*}
    Risk = Impact \cdot Probability
\end{equation*}
\textsc{CSC} recognized that without a substantial quantitative update, \climatedge could not address the probability of events occurring. \textsc{CSC} then made a strategic decision to focus the second iteration of \climatedge on quantitative prediction. As with the original version, it offers a view into climate prediction based on proprietary analysis of public data. In 2011, the U.S. premiums for property and casualty insurance totaled \$442 billion \cite{iii}. Opportunities exist within this overall market space that \climatedge cannot address without a significant update in how the data is stored, and subsequently analyzed.
\subsection{Hypothesis}
Without investing, developing, and applying expertise in big data \index{big data} storage and analytic technologies, future \climatedge offerings will be limited to markets served by processing relatively small and straightforward data sets. By using technologies such as virtualization, distributed databases, elastic compute and storage, and distributed algorithms, \textsc{CSC} can quantitatively analyze more data and offer \climatedge reporting to an ever increasing number of industries.
\subsection{Offerings}
\textsc{CSC's} Data Services group, in conjunction with industry analysts and other internal groups,  has identified several business cases  in which an updated \climatedge offering can benefit the general insurance industry [personal communication, 2013]:
\begin{itemize}
    \item tornado prediction
    \item flood prediction
    \item hail prediction \& forensics
\end{itemize}
 This paper selects the tornado and flood predictive analytics and views them through the lens of big data.
