%!TEX root=index.tex
\section{Executive Summary}
\textsc{CSC's} \climatedge offering is focused on providing predictive climate analytics using public government data sets. By using various algorithms, both proprietary and public, \textsc{CSC} hopes to offer its customers a distinct advantage by allowing strategic business decisions to be made based on forward looking climate forecasts. Planned offerings for the general insurance industry include tornado count prediction, flood prediction, hail prediction and forensics.\\

This paper focuses on contrasting the tornado and flood offerings with respect to the technology needed to store, process, and present the analytics. Tornado prediction involves running a published algorithm against a relatively small amount of well formed data on any modern home computing system. Flood prediction, on the other hand, requires substantially larger and more diverse data such as newspaper reports and building plans. The same platform which delivered the tornado counts is not capable of handling flood prediction. Without an investment in big data technology and expertise, the range of \climatedge offerings will be limited.\\

Following that, a reference framework implementation is presented that will allow future \climatedge offerings to scale in processing both large and complex data sets, thus enabling a wider variety of business cases to be undertaken. The most important characteristic is flexibility in technologies and a loose coupling between framework layers. There does not exist a single application capable of addressing every offering scenario, so a collection of technologies that can be integrated together is presented. As technologies evolve, the goal is to be able to replace any one piece of technology without overly affecting other layers.

Finally, future offerings are considered and presented, along with various impacts these  will have on the reference framework.