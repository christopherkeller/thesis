%!TEX root=index.tex
\section{Appendix}
\newfontfamily{\anonymous}[Scale=MatchLowercase]{Anonymous Pro}
\lstset{ %
    language=C,                             % Code langugage
    basicstyle=\anonymous,
    tabsize=1,
    breaklines=true,
    breakatwhitespace=false,
    showstringspaces=false,
    showspaces=false,
    showtabs=false
}
Listed below is C code which processes a \gls{merra} data file and stores it within a simple SQLite database \cite{keller1}. Every data type would have a similar parser, typically written in Python for ease of development. However given the complex structure of the \gls{merra} format, combined with the size of the data set, it is reasonable to spend the time to develop it in C to gain execution speed. Processing a single day of \gls{merra} (300MB) and writing into an SQLite flat file takes approximately 20 seconds per variable with the listed code on a modern CPU. Similar code in Python was taking several hours to complete. The entire thirty-three year \gls{merra} archive would take approximately one week to ingest, running in parallel. 
 
\lstinputlisting{/Users/christopher/Documents/work/Climatalytics/prototype/source/merra/c/geo_point.h}
\hrule
\lstinputlisting{/Users/christopher/Documents/work/Climatalytics/prototype/source/merra/c/merra_regex.h}
\hrule
\lstinputlisting{/Users/christopher/Documents/work/Climatalytics/prototype/source/merra/c/merra_regex.c}
\hrule
\lstinputlisting{/Users/christopher/Documents/work/Climatalytics/prototype/source/merra/c/sql.h}
\hrule
\lstinputlisting{/Users/christopher/Documents/work/Climatalytics/prototype/source/merra/c/sql.c}
\hrule
\lstinputlisting{/Users/christopher/Documents/work/Climatalytics/prototype/source/merra/c/latlon.c}
\hrule
\lstinputlisting{/Users/christopher/Documents/work/Climatalytics/prototype/source/merra/c/main.c}
\lstset{ %
    language=Python,                             % Code langugage
    basicstyle=\anonymous,
    tabsize=1,
    breaklines=true,
    breakatwhitespace=false,
    showstringspaces=false,
    showspaces=false,
    showtabs=false
}
A simple mapreduce implementation for word counting in Python is presented below \cite{keller2}.
\lstinputlisting{/Users/christopher/Development/personal/mapreduce_python/pr.py}
